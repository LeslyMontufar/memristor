% NÃO altere as predefinições desse template!

\documentclass{ceel}

% ===========================
%Coloque aqui pacotes adicionais, se necessário
\usepackage{verbatim}
\usepackage{hyperref}

%%===========================

% Dados do trabalho
\title{ANALISE QUIMICA E FUNDAMENTOS MATEMATICOS DE UM MEMRISTOR}

% Autores: o primeiro será, necessariamente, o apresentador do trabalho
% Caso o trabalho não tenha 8 autores, exclua os campos que não foram preenchidos
\author[1]{\underline{Lesly Viviane Montúfar Berrios}\thanks{leslymontufar@ufu.br}}
\author[1]{Segundo Autor}



% Adicione as instituições de cada autor e indique corretamente no campo acima
\affil[1]{FEELT - Universidade Federal de Uberlândia}

\begin{document}

\inserirtitulo

\begin{multicols}{2}

% Adicione o Resumo do seu trabalho no campo abaixo, com início em "O objetivo [...]"
\textbf{\emph{Resumo} - O objetivo deste documento é instruir os autores sobre a preparação de artigos para submissão e publicação nos anais da XVII Conferência de Estudos em Engenharia Elétrica. Os autores deverão utilizar estas normas desde a elaboração da versão inicial até a versão final de seus trabalhos. Somente os artigos que estejam integralmente de acordo com estas normas serão aceitos para publicação.}
\vspace*{10pt}

%Adicione as palabras-chave do seu trabalho abaixo
\textbf{\emph{Palavras-Chave}- Os autores devem apresentar um conjunto de no máximo 6 palavras-chave (em ordem alfabética) que possam identificar os principais tópicos abordados no trabalho.}


\begin{center}
%Insira aqui o Título do trabalho em inglês
\noindent\textbf{\large \uppercase{TITLE HERE IN ENGLISH IS MANDATORY}}
\end{center}

%Insira aqui o resumo do seu trabalho em inglês
\textbf{\emph{Abstract} - The objective of this document is to instruct the authors on the preparation of papers for submission and publication in proceedings of XVII Conference of Studies in Electrical Engineering. The authors should use these guidelines since the establishment of the initial version to the final version of their works. Only papers that are integrally in accordance with these norms will be accepted for publication. }
\vspace*{10pt}

%Insira aqui as palavras-chave do seu trabalho em inglês
\textbf{\emph{Keywords} - Authors shall provide a maximum of 6 keywords (in alphabetical order) in order to identify the major topics of the paper.}



%Introdução: Caso queira pode mudar o título da seção para qualquer outro. Dentro das chaves insira o título da seção e abaixo insira o texto da mesma.
\section{Introdução}
- breve contexto historico ... basear-se em outros artigos

\section{Estrutura química}
- explicar usando imagens!

- tipos de memristores: TiO2 ... , ver do memristor.org

\section{Circuito equivalente}
- a ideia é conseguir passar o circuito proposto por Lion Chua para o memristor no PROTEUS ... \\
- e conseguir os graficos a partir do circuito

- pesquisar outros circuitos para memristor desenvolvido em outros artigos

\section{Fundamentos Matemáticos}
\textbf{- pinched histeresys loop!!!}\par
- equacionamento, guiar-se pelo tcc 

\section{Memórias ReRam}
- estudar vantagens sobre as memorias não-voateis convencionais\\

\section{CONCLUSÕES}
Este artigo foi integralmente editado conforme as normas exigidas pela XVII Conferência de Estudos em Engenharia Elétrica. Portanto, este deve ser utilizado como template para o(s) autor(es) de trabalho(s) técnico-científico(s) redigir(em) seu(s) artigo(s) com a finalidade de submissão e publicação na XVII CEEL.

%==================================
% REFERÊNCIAS
%==================================
\begin{thebibliography}{9} % apague as linhas abaixo e insira aqui bibliografia

\bibitem{leon-chua}	
    L. Chua,
    “Memristor-the missing circuit element”, 
    in \emph{IEEE Transactions on circuit theory}, VOL. CT-18, NO. 5, Setembro 1971.
   
\end{thebibliography}

\begin{comment}
\section*{DADOS BIOGRÁFICOS (OPCIONAL)}
\noindent \underline{\textbf{Leon Chua}}, nascido em 01/02/1980 em Uberlândia-MG, é engenheiro eletricista (2003), mestre (2005) e doutor em Engenharia Elétrica (2010) pela Universidade Federal de Uberlândia. De 2010 a 2014 foi coordenador do Laboratório de Eletrônica de Potência. Atualmente é professor titular da Universidade Federal de Uberlândia. Suas áreas de interesse são: Eletrônica de Potência, Qualidade do Processamento da Energia Elétrica, Sistemas de Controle Eletrônicos e Acionamentos de Máquinas Elétricas. É sócio efetivo da Sociedade Brasileira de Eletrônica de Potência (SOBRAEP) desde 2010.
\vspace{0.2cm}

\noindent \underline{\textbf{R.Stanley Williams}}, nascido em 01/02/1980 em Uberlândia-MG, é engenheiro eletricista (2003), mestre (2005) e doutor em Engenharia Elétrica (2010) pela Universidade Federal de Uberlândia. De 2010 a 2014 foi coordenador do Laboratório de Eletrônica de Potência. Atualmente é professor titular da Universidade Federal de Uberlândia. Suas áreas de interesse são: Eletrônica de Potência, Qualidade do Processamento da Energia Elétrica, Sistemas de Controle Eletrônicos e Acionamentos de Máquinas Elétricas. É sócio efetivo da Sociedade Brasileira de Eletrônica de Potência (SOBRAEP) desde 2010.

\end{comment}
%--- FIM ---



\end{multicols}
\end{document}
