\documentclass{ceel}

% ===========================
%Coloque aqui pacotes adicionais, se necessário
\usepackage{hyperref}
\usepackage{verbatim}
\usepackage{array}
\usepackage{amsmath}
\usepackage{subcaption, float}
%%===========================

% Dados do trabalho
\title{Fundamentos Físico-químicos e Matemáticos de um Memristor}

%\author[1]{\underline{Lesly Viviane Montúfar Berrios}\thanks{leslymontufar@ufu.br}}
%\author[1]{Cibelly Cristina Rodrigues Couto\thanks{cibelly.cris@ufu.br}}
%\author[1]{Yasmin Delbany Cury\thanks{yasmin.cury@ufu.br}}
%\author[2]{Paulo Henrique Oliveira Rezende\thanks{paulohenrique.rezende@ufu.br}}

\author[1]{\underline{Nome Autor Apresentador}\thanks{apresentador@ufu.br}}
\author[1]{Nome do Segundo Autor\thanks{segundoautor@ufu.br}}
\author[1]{Nome do Terceiro Autor \thanks{terceiroautor@ufu.br}}
\author[2]{Nome do Autor Orientador\thanks{orientador@ufu.br}}


\affil[1]{FEELT - Universidade Federal de Uberlândia}
\affil[2]{FEELT - Professor Adjunto - Universidade Federal de Uberlândia}

\begin{document}

\inserirtitulo

\begin{multicols}{2}

\textbf{\emph{Resumo} - Um estudo sobre o comportamento do quarto elemento de circuito fundamental idealizado por Leon Chua em 1971 e implementado primeiramente pela equipe da \emph{HP Labs} liderada por R. Stanley Williams em 2008 é apresentado. Chamado de \emph{memristor}, o novo elemento de circuito passivo de duplo terminal relaciona as variáveis de carga $q(t)=\int_{-\infty}^t i(\tau)\, d\tau$ e fluxo $\varphi(t)=\int_{-\infty}^t v(\tau)\, d\tau$ e comporta-se como um resistor não-linear com memória. Sua propriedade peculiar advém da capacidade do material de manter seu último estado, e permite aplicações em diversos contextos, como em memórias ReRam e computação neuromórfica.} %% aplicacoes a completar
\vspace*{10pt}

\textbf{\emph{Palavras-Chave} - Aplicações, Físico-Químico, Matemática, Memristor.}


\begin{center}

%Insira aqui o Título do trabalho em inglês
\noindent\textbf{\large \uppercase{Physicochemical and Mathematical Foundations of a Memristor}}
\end{center}

\textbf{\emph{Abstract} - A study of the behavior of the fourth fundamental circuit element devised by Leon Chua in 1971 and first implemented by the \emph{HP Labs} team led by R. Stanley Williams in 2008 is presented. Called \emph {memristor}, the new two-terminal passive circuit element lists the variables of charge $q(t)=\int_{-\infty}^t i(\tau)\, d\tau$ and flux $\varphi(t)=\int_{-\infty}^t v(\tau)\, d\tau$ and behaves like a nonlinear resistor with memory. Its peculiar property comes from the material's ability to maintain its last state, and allows applications in various contexts, such as ReRam memories and neuromorphic computing.}
\vspace*{10pt}

\textbf{\emph{Keywords} - Applications, Mathematics, Memristor, Physicochemical.}


\section{Introdução}
A teoria de circuitos elétricos, há 150 anos, abrangia basicamente três componentes passivos fundamentais: o capacitor (1745), o resistor (1827) e o indutor (1831). No entanto, em 1971, o professor phD da Universidade da Califórnia, Leon Chua, apresentou novas considerações a partir da análise das possíveis combinações entre as quatro variáveis fundamentais de circuitos: corrente elétrica $i$, tensão elétrica $v$, carga elétrica $q$ e fluxo magnético $\varphi$ – sendo as duas últimas descritas como integrais no tempo da corrente, $q(t)=\int_{-\infty}^t i(\tau)\, d\tau$, e da tensão, $\varphi(t)=\int_{-\infty}^t v(\tau)\, d\tau$, respectivamente.

Chua observou que o capacitor é definido pela relação entre carga $q(t)$ e tensão $v(t)$ via $qd=C dv$. Similarmente, o resistor pela relação entre corrente $i(t)$ e tensão $v(t)$ via $dv=R di$, e o indutor pela relação entre fluxo magnético $\varphi(t)$ e corrente $i(t)$ via $d\varphi(t)=L di$. Teria-se então que a combinação das quatro variáveis fundamentais de circuitos resultaria em somente três componentes fundamentais.
% passivos.
Desse modo, baseando-se no argumento da simetria, o estudioso postulou que haveria um elemento de circuito faltante, capaz de associar a carga $q(t)$ e o fluxo magnético $\varphi(t)$, o que o levou, em 1971, a publicar um artigo no qual idealiza o novo componente, definido pela relação $d\varphi=M dq$ e que denominou \emph{memristor}, uma contração de \emph{memory resistor} \cite{artigo}. 

%, em português \emph{resistor com memória} 
Portanto, considerado como o quarto elemento fundamental dos circuitos eletrônicos, ao lado do capacitor, resistor e indutor, o \emph{memristor} destaca-se por apresentar uma propriedade peculiar, cuja explanação e abordagem teórica é tratada adiante. É definido, assim como um resistor, como um componente eletrônico passivo de duplo terminal, utilizado para limitar a corrente em um circuito e dissipar energia térmica, com o diferencial de que, para o \emph{memristor}, essa limitação, chamada de resistência ou impedância, alterar-se conforme a quantidade de carga elétrica que flui em si e mantém o valor da última resistência obtida até a aplicação de nova carga.

Apesar da proposta teórica do \emph{memristor} ter sido apresentada por Chua em 1971, sua primeira implementação prática ocorreu apenas em 2008, nos \emph{Laboratórios da Hewllet-Packard} (\emph{HP}), graças à equipe liderada pelo físico-químico Dr. Richard Stanley Williams, que desenvolveu linhas de memristores com base no dióxido de titânio ($TiO_2$) em escala nanométrica \cite{nature}. A demora deve-se, principalmente, à dificuldade em encontrar materiais que fossem capazes de conferir a propriedade de reter memória ao dispositivo eletrônico e que satisfizessem a base teórica apresentada por Chua \cite{artigo}.
Dessa forma, o componente recentemente sintetizado contém a propriedade da não-volatilidade, que, aliada a possibilidade de ser trabalhado em escala nanométrica, o torna promissor em aplicações e garante sua contribuição na validação da Lei de Moore.

Pissardini \cite{memcomputacao} atenta para aplicações de \emph{memelementos} na elaboração de novas arquiteturas computacionais, que, corroborado pelos ideais da Arquitetura de Von Neumann, são capazes de realizar tarefas específicas. Essa abordagem tem sido conhecida como \emph{memcomputação}, que, ao contrário dos modelos convencionais, propõe que seja possível processar e armazenar dados no mesmo dispositivo (elemento físico), o qual seria o memristor em si. Prezioso \cite{rneurais} propõe que existem também aplicações de memristores e de memelementos nas chamadas redes neurais, as quais são sistemas que se baseiam em princípios de Inteligência Artificial, podendo aprender e reconhecer padrões, e que podem ser aprimoradas com o uso da memresistência devido ao caráter de não-volatilidade dos memristores. Dong e Qi \cite{chines}, em seu artigo acerca de circuitos lógicos, analisam a importância de memristores para otimizar
seu funcionamento, devido à característica não linear que possuem.

Sob essa perspectiva, a temática deste artigo basear-se-á na descrição detalhada desse novo componente, nos âmbito físico, químico e matemático, com intuito de compreender como o mecanismo interno converge para as propriedades impostas,  por Chua, a um dispositivo \emph{memristor}. É de interesse ainda discutir acerca das aplicabilidades do dispositivo, para assim poder expor o impacto de sua descoberta na teoria de circuitos e importância para o crescimento tecnológico. %% eletronico com SPICE
%% se formos acrescentar mais alguma coisa como circuito LTSpice.
%% depende das aplicacoes

Na Seção \ref{estrutura} são apresentadas as características estruturais, físico-químicas, de um \emph{memristor}, com ênfase para o modelo da \emph{HP Labs}. Para, na Seção \ref{analise-matematica}, analisá-lo matematicamente e, assim, provar a natureza de suas propriedades e peculiaridades a partir do equaciomento e disposição de gráficos. Além disso, a Seção \ref{sim} apresenta simulações computacionais no ambiente \emph{MATLAB}
% e \emph{LTSPICE} 
que exemplificam suas características marcantes. Finalmente, na Seção \ref{aplicacoes}, é explicitada algumas de suas aplicações e recentes estudos.


%% precisa colocar imagem pequena para explicar que a propriedade do memristor deve-se basicamente à redistribuição das lacunas(+) devido à deficiencia em oxigenio.
%% de preferencia a msm imagem do R. Stanley Williams
\section{Características estruturais de um memristor} \label{estrutura}
Dispositivos de resistência variável baseados em óxidos 
possuem vasta aplicabilidade, devido à capacidade de reter memória, pois
admitem características de alta velocidade, alta densidade e operação em baixa energia, 
assim como a não-volatilidade. Desse modo, os materiais óxidos são ideiais para implementar o conceito de um \emph{memristor}, sendo recentemente utilizados para esse fim os óxidos de titânio e tântalo ($TiO_x$, $TaO_x$) \cite{conceito}. %% Física do estado sólido

Para analisar o funcionamento de um memristor, é interessante compreender o dispositivo \emph{crossbarlatch}, estudado em 2006 pela
\emph{HP Labs}, posteriormente passou a ser conhecido como o próprio memristor. O crossbarlatch era uma espécie de sanduíche, possuindo partes externas constituídas por Platina ($Pt$), com espessura cerca de 3 nanômetros, e uma camada intermediária isolante subdividida em duas partes, sendo a primeira formada de óxido de titânio ($TiO_2$), conhecida como parte pura ou não dopada,e a segunda pela mesma substânca porém deficiente em oxigênio ($TiO_{2-x}$), conhecida como parte dopada \cite{construcao}.  Nesse sentido, o funcionamento molecular de um memristor é baseado no movimento de íons livres, no meio dielétrico (óxido), o que torna possível o comportamento memristivo, pois cria mudanças locais na condutividade a partir da modelagem de filamentos condutores entre os eletrodos.

Esses filamentos condutores entre os eletrodos no meio inicialmente dielétrico são resultantes de um processo conhecido como \emph{eletroformação}, do inglês \emph{electroforming}, que consiste na aplicação de uma tensão elétrica suficientemente alta, ou seja um valor limite de tensão, entre os dois eletrodos por um período de tempo suficiente para a formação de um canal condutivo local no meio isolante, o que significa o rompimento de sua rigidez dielétrica.
Ademais, esses valores limite de tensão e de tempo necessários para a formação do canal condutivo dependem dos materiais que constituem os dielétricos e os eletrodos, além da estrutura geométrica do dispositivo \cite{us}. 

\vspace{-0.26cm}
\begin{figure}[H]
\centering
\includegraphics[width=\columnwidth]{oxygen-vacancies}
\caption{(a) Distribuição de vacâncias de oxigênio na camada intermediária de um memristor. (b) Aplicando-se tensão positiva os íons livres são repelidos. (b) Aplicando-se tensão negativa os íons livres são atraídos.}\label{estrutura}
\end{figure}

Devido à flexibilidade de variação de valores de resistência, ao memristor podem ser atribuídos \emph{status} como “OFF”, quando a parte não-dopada é dominante portanto o estado é de alta resistência; “ON”, quando a parte dopada é dominante portanto há baixa resistência, e ainda estados intermediários. Assim, pode armazenar dados na forma binária e multinível, como ocorre na Figura \ref{estrutura}(a). As nanopartículas utilizadas na dopagem possuem carga $+2$ (vacâncias de oxigênio) como é representado, e percebe-se uma proporção igualitária entre parte dopada e não dopada. 

Observa-se ainda que, a partir de um estado inicial qualuer como o da Figura \ref{estrutura}(a), ao se aplicar uma tensão elétrica positiva 
no terminal dopado, as lacunas (vacâncias de oxigênio), que comportam-se como cargas positivas, migram para o lado oposto, ou seja, cargas negativas se deslocam para a parte dopada.
 Dessa forma, a espessura da camada intermediária não-dopada diminui, ao mesmo tempo que a espessura da camada dopada e de maior condutividade aumenta, o que resulta na diminuição do valor da resistência do dispositivo. Tal estado do memristor é conhecido como "ON", ou de baixa resistência, e é ilustrado na Figura \ref{estrutura}(b). 
%
Por outro lado, ao se aplicar uma tensão negativa no terminal dopado, as lacunas são atraídas, ou seja, a espessura da camada intermediária não-dopada aumenta, ao mesmo tempo que a espessura 
da camada de maior condutividade diminui, causando um aumento no valor da resistência do memristor. Tal estado, ilustrado na Figura \ref{estrutura}(c), é conhecido como "OFF", ou de alta resistência. 

Assim, o princípio básico do funcionamento de um memristor consiste basicamente na redistribuição das lacuna sob a influência de uma tensão elétrica aplicada e, quando cessada, as 
camadas pura e dopada permanecem como no último estado atingido, dessa forma memorizam o último valor de resistência alcançado, e esse processo é facilitado por ocorrer em 
escala nanométrica. %% por que? equacao 1/D^2

\section{Fundamentos Matemáticos}\label{analise-matematica}
O \emph{memristor}, a princípio, em 1971, foi definido por Chua \cite{artigo} como elemento de circuito de duplo-terminal caracterizado pela relação do tipo $g(\varphi, q)=0$. Diz-se que 
um memristor é controlado por carga se a relação entre fluxo e carga é expressa como uma função de carga elétrica $q$, e diz-se que é controlado por fluxo se é expressa em função do 
fluxo $\varphi$. Para um memristor controlado por carga tem-se a Equação (\ref{mem-charge}) e derivando-se ambos os termos consegue-se a Equação (\ref{derivate-q}).

\begin{equation}\label{mem-charge}
\varphi = f(q)
\end{equation}

\begin{equation}\label{derivate-q}
\dfrac{d\varphi}{dt}=\dfrac{df(q)}{dq} \ \dfrac{dq}{dt}
\end{equation}
\vspace{0.1cm}

Sabendo-se ainda que $v(t)=\dfrac{d\varphi}{dt}$ e $i(t)=\dfrac{dq}{dt}$ descrevem, respectivamente, a tensão e corrente elétrica no memristor, 
reescreve-se a Equação (\ref{derivate-q}) como na Equação (\ref{M}) 

\begin{gather}\label{vMi}
v(t)=M(q)\ i(t)
\end{gather}

\noindent onde
\begin{equation} \label{M}
M(q) =\dfrac{df(q)}{dq}
\end{equation}
\vspace{0.05cm}

$M$ é chamada de \textit{memristência}, e é mensurada na mesma unidade que a resistência (\textit{Ohms} - $\Omega$). A memristência, definida pela Equação (\ref{M}), define uma relação linear entre tensão e corrente, enquanto a carga for constante. Logo, se $M$ é constante, o memristor comporta-se como um resistor.

Para um memristor controlado por fluxo tem-se a Equação (\ref{mem-flux}) e derivando-se ambos os termos, a Equação (\ref{derivate-phi}).

\begin{equation}\label{mem-flux}
q = f(\varphi)
\end{equation}

\begin{equation}\label{derivate-phi}
\dfrac{dq}{dt}=\dfrac{df(\varphi)}{d\varphi} \ \dfrac{d\varphi}{dt}
\end{equation}
\vspace{0.05cm}

Reescrevendo a Equação (\ref{derivate-phi}) utilizando-se as definições de tensão e corrente tem-se a  Equação (\ref{iWv}).

\begin{gather}\label{iWv}
i(t)=W(\varphi)\ v(t)
\end{gather}

\noindent onde
\begin{equation} \label{W}
W(\varphi) =\dfrac{df(\varphi)}{d\varphi}
\end{equation}
\vspace{0.05cm}

Nesse caso, $W$, da Equação \ref{W}, é chamado de \textit{memductância} e tem mesma unidade da condutância (\textit{Siemens} - $S$).

Entretanto, em especial para o memristor controlado por por carga, as relações das Equações (\ref{vMi}) e (\ref{vMi}) não abrangem a propriedade de memória do dispositivo, o que levou Chua juntamente com seu aluno de graduação Kang \cite{1976}, em 1976, a reconhecer que, enfaticamente, o \emph{memristor} seria um componente passivo com \emph{estado}, ou seja, cuja propriedade de memória é definida por uma ou mais \emph{variáveis de estado}.
Diante disso, a equipe HP Laboratories modelou matematicamente um \emph{memristor}, essencialmente, a partir das equações propastas por Chua e Kang \cite{1976} para sistemas memristivos (Equações \ref{sm1} e \ref{sm2}).

\begin{comment}
\begin{equation}\label{eq01}
v=M( w) \ i
\end{equation}
\begin{equation}\label{eq02}
\dfrac{dw}{dt} =f(i) 
\end{equation}
\end{comment}

Assim, a definição matemática rigorosa revela que, conforme a Equação (\ref{eq01}), a \emph{memristência} $M$ depende do estado do dispositivo, em determinado instante de tempo $t$, que por sua vez também varia com o tempo como integral da função contínua de corrente $f$, como descrito pela Equação (\ref{eq02}).  Uma representação física da variável de estado $w$ é ilustrada na Figura \ref{w}. 

Sistemas memristivos Chua e Kang:
\begin{equation}\label{sm1}
v=M(w, i)i
\end{equation}

\begin{equation}\label{sm2}
\dfrac{dw}{dt}=f(w, i)
\end{equation}


\subsection{Modelo de Deriva Linear}
Continuacao:
\begin{equation}\label{}
M(w)=\dfrac{w}{D}R_{ON}+\Big(1 - \dfrac{w}{D}\Big) R_{OFF}
\end{equation}

\begin{equation}\label{}
\dfrac{dw}{dt}=v_D \text{given by}=\dfrac{\mu_D R_{ON}}{D}i(t)
\end{equation}

Integrando-se: divide por D...
\begin{equation}\label{}
\dfrac{w(t)}{D}=\dfrac{w_0}{D}+\dfrac{\mu_D R_{OFF}}{D^2}q(t)
\end{equation}

\begin{equation}\label{}
\dfrac{w(t)}{D}=\dfrac{w_0}{D}+\dfrac{q(t)}{Q_D}
\end{equation}
\begin{equation}\label{}
x(t)=x(t_0)+\dfrac{q(t)}{Q_D}
\end{equation}

Agora a variavel de estado x:
\begin{equation}\label{}
M(t)=R_{ON}\ x(t)+ R_OFF(1-x(t))
\end{equation}

t=0 inicio do gather!!!
\begin{align}\label{}
M_0&=R_{ON}\ x(t_0)+ R_OFF(1-x(t_0))&\\
    &=R_{ON}\Bigg( x(t_0)+ \dfrac{R_{OFF}}{R_{ON}} (1-x(t_0))\Bigg)&\\
    &=R_{ON}\Bigg( x(t_0)+ \dfrac{R_{OFF}}{R_{ON}} (1-x(t_0))\Bigg)&\\
    &=R_{ON}( x(t_0)+ r (1-x(t_0)))&
\end{align}

Substitui x(t) em M(x) = ...
\begin{equation}\label{}% exclui essa talvez
M(q)=R_{OFF}\Bigg( 1-\dfrac{q(t)}{Q_D}\Bigg)
\end{equation}

\begin{equation}\label{}% essa direto
M(q)=M_0-\Delta R\Bigg(\dfrac{q(t)}{Q_D}\Bigg)
\end{equation}

Substitui na relacao geral:
\begin{equation}\label{}
v(t)=\bigg(M_0-\Delta R\Big(\dfrac{q(t)}{Q_D}\Big)\bigg)\dfrac{dq}{dt} = \dfrac{d}{dt} \bigg(M_0 q-\dfrac{\Delta R\ q^2(t)}{2 Q_D}\bigg)
\end{equation}

Integra e báskara e usa M=dphi/dq:
\begin{equation}\label{}
q(t)=\dfrac{Q_D\ M_0}{\Delta R}\Bigg(1\pm\sqrt{1-\dfrac{2\, \Delta R}{Q_D\ M^2_0}\varphi(t)}\Bigg)
\end{equation} 

Aproximacoes:
\begin{equation}\label{}
q(t)=Q_D\Bigg(1-\sqrt{1-\dfrac{2}{Q_D\ R_{OFF}}\varphi(t)}\Bigg)
\end{equation} 

Estado inrerno do memristor:
% ... operações com align
\begin{equation}\label{}
x(t)=1-\sqrt{1-\dfrac{2\mu_D}{r\; D^2}\varphi(t)}
\end{equation} 

Assim: escala nanometrica!
\begin{equation}\label{}
M(q)=R_{OFF}\Bigg(\sqrt{1-\dfrac{2\mu_D}{r\; D^2}\varphi(t) }\Bigg)
\end{equation}

\begin{equation}\label{}
i(t)=\dfrac{v(t)}{R_{OFF}\Bigg(\sqrt{1-\dfrac{2\mu_D}{r\; D^2}\varphi(t) }\Bigg)}
\end{equation} 



\section{Simulações} \label{sim}

\begin{figure}[H]
\centering

\begin{subfigure}{0.49\columnwidth}
\centering
\includegraphics[width=\columnwidth]{flux-charge}
\caption{Left figure} \label{fig:left}
\end{subfigure}
\hfill
\begin{subfigure}{0.49\columnwidth}
\centering
\includegraphics[width=\columnwidth]{pinched-hysteresis-loop}
\caption{Right figure} \label{fig:right}
\end{subfigure}

\caption{}\label{}
\end{figure}

%% Lisajous Figures



\section{Aplicações}\label{aplicacoes}
\subsection{Mémorias ReRam}
A Memória Resistiva de Acesso Randômico (ReRAM), assim como a memória de acesso aleatório magnéticas (MRAM), mantêm os dados na falta de energia, essa propriedade deve-se especialmente pela presença do memristor em seus chips, que altera sua resistência à passagem da corrente elétrica em resposta a variações na sua tensão de alimentação.
Essa tecnologia assemelha ao uso de transistores nas memórias Flash, com a diferença de as células de memória ReRAM poderem ser muito menores, consumir menos energia elétrica e ter desempenho para leitura e escrita de dados próximos aos de módulos da Memória de Acesso Randômico Dinâmica (DRAM)\cite{blog} , que diferentemente da memória ReRAM (que armazena os dados como resistência), essa armazena os dados como carga elétrica.
O reRAM, portanto, tem o potencial de ser extremamente denso, de baixa potência, com alta resistência, tornando-se uma tecnologia atraente para armazenamento secundário e níveis de memória de classe de armazenamento \cite{prog}. Segundo a HP, dispositivos contendo memristores oferecem em média algo em torno de 20GB por centímetro quadrado, o dobro da capacidade projetada para as memórias flash \cite{hp}.
Dispositivos memresistivos podem mudar o paradigma padrão da computação ao permitir que cálculos sejam executados nos chips onde a informação está armazenada \cite{hp}. Além disso, essa tecnologia é passível de utilização em todo o tipo de dispositivo, desde celulares e MP3 players, que primariamente utilizam memórias flash NAND, como SSD (solid state disk), as memórias flash e DRAM.

\subsection{Circuitos Lógicos} %  logic, hybrid circuits with CMOS, and neuromorphic computing.


\section{CONCLUSÕES}


%==================================
% REFERÊNCIAS
%==================================
\begin{thebibliography}{9}

%% INTRODUCAO
\bibitem{artigo}	
    L. Chua,
    “Memristor - the missing circuit element”, 
    in \emph{IEEE Transactions on circuit theory}, VOL. CT-18, NO. 5, Setembro 1971.

\bibitem{nature}	
     D.B. Strukov, G.S. Snider, D.R. Stewart,  R.S Williams, 
"The missing memristor found", 
\emph{Nature}, 2008, 1 May 2008, vol. 453, pp. 80-83.

\bibitem{memcomputacao}
    R. S. Pissardini,
    “MEMCOMPUTAÇÃO: CARACTERÍSTICAS E APLICAÇÕES EM
COMPUTAÇÃO PARALELA”, Escola Politécnica da Universidade de São Paulo, Departamento de Engenharia de Transportes, Laboratório de Topografia e Geodesia.
 Disponível em:
 \url{https://www.ime.usp.br/~gold/cursos/2015/MAC5742/reports/MemComputacao.pdf}. Acesso em: jun. 2019.
 
 \bibitem{rneurais}	
   M. Prezioso, F. Merrikh-Bayat, B. D. Hoskins, G. C. Adam, K. K. Likharev, D. B. StrukovTraining, "\uppercase{Training and operation of anintegrated neuromorphic
network based on metal-oxide memristors}", 
Department of Electrical and Computer Engineering, University of California at Santa Barbara, Santa Barbara, California 93106, USA. 
Department of Physics and Astronomy, Stony Brook University,
Stony Brook, New York 11794, USA. Disponível em: https://www.ece.ucsb.edu/~strukov/papers/2015/Nature2015.pdf. Acesso em 13/08/2019

\bibitem{chines}
    Z. Dong, D. Qi, Y. He, Z. Xu, X. Hu, S. Duan, "\uppercase{Easily Cascaded Memristor-CMOS Hybrid Circuit
for High-Efficiency Boolean Logic Implementation}", School of Computer and Information Science,
College of Electronics and Information Engineering,
Southwest University, Chongqing 400715, P. R. China . Disponível em:\url{https://www.researchgate.net/publication/329039458\_Easily\_Cascaded\_Memristor-CMOS\_Hybrid\_Circuit\_for\_High-Efficiency\_Boolean\_Logic\_Implementation}. Acesso em ago. 2019.

\begin{comment}
\bibitem{cibelly1}	
    D. Biolek, Z. Biolek, and V. Biolková, “\uppercase{SPICE modeling of memristive,
    memcapacitative and meminductive systems},” in Proc. of the European
    Conference on Circuit Theory and Design (ECCTD09), Antalya,
    Turkey, 2009, pp. 249-252. %% ainda não
\end{comment}

%% FUNCIONAMENTO ESTRUTURAL
\bibitem{conceito}	
    J. P. Strachan, A. C. Torrezan, F. Miao, M. D. Pickett, J. J. Yang; W. Yi, G. M. Ribeiro, R. S. Williams, “STATE DYNAMICS AND MODELING OF TANTALUM OXIDE MEMRISTORS”, Disponível em: \url{https://ieeexplore.ieee.org/stamp/stamp.jsp?arnumber=6542012}. Acesso em ago. 2019.
 
\bibitem{construcao}	
   F.S.Barachati, "ESTUDO E PREPARAÇÃO DE MEMORISTORES PARA SUA APLICAÇÃO EM DISPOSITIVOS ELETRÔNICOS". Escola de Engenharia de São Carlos, da Universidade de São Paulo. Disponível em: \url{http://www.tcc.sc.usp.br/tce/disponiveis/18/180450/tce-26032012-090929/publico/Barachati_Fabio_Souza.pdf}. 
   Acesso em ago. 2019.   

\bibitem{us}	
    United States Patent; Patent No: US 9035272 B2. “NANOPARTICLE-BASED MEMRISTOR STRUCTURE”. Disponível em: \url{https://patentimages.storage.googleapis.com/0e/70/e0/6b8926f49e7cd8/US9035272.pdf}. Acesso em ago. 2019.
    

%% MATEMATICA
\bibitem{elusive} % indescritivel
    Y. N. Joglekar, S. J. Wolf, "The elusive memristor: properties of basic electrical circuits", Department of Physics, Indiana University Purdue, 2009.

\bibitem{1976}	
    L. Chua, S. M. Kang,
    “Memristive devices and systems”, 
    in \emph{Proceedings IEEE}, 1976, vol. CT-18, no. 5, pp. 507-519.

%% APLICACOES

%% MEMORIAS RERAM
\bibitem{blog}
   E. Alecrim, "Os primeiros chips ReRAM fabricados em larga escala chegarão em breve", 2013. Disponível em: \url{https://tecnoblog.net/136874/primeiros-chips-reram-em-larga-escala/}. Acesso em ago. 2019.
   
\bibitem{prog}
    M. Ramadan, N. Wainstein, R. Ginosar, S. Kvatinsky, "Adaptive programming in multi-level cell ReRAM", Microelectronics Journal, Volume 90, 2019. %% MULTI-LEVEL CELL == MLC ?!
    %% bem recente :)
    
\bibitem{hp}
A. Martins, "ReRAM - A próxima geração de memórias e CPU’s", 2010. Disponível em: \url{https://brainstormdeti.wordpress.com/2010/09/15/reram-\%E2\%80\%93-a-proxima-geracao-de-memorias-e-cpu\%E2\%80\%99s/}. Acesso em ago.2019

%% CIRCUITOS LOGICOS


\end{thebibliography}


\end{multicols}
\end{document}
